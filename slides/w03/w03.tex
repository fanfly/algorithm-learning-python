\documentclass{beamer}
\usepackage{../mysty}

\title{Week 03: Selection}
\author{FanFly}
\date{March 22, 2020}

\begin{document}

\begin{frame}
  \titlepage
\end{frame}

\section{The Selection Problem}
\subsection{Order Statistics}
\begin{frame}{Order Statistics}
  First, let us introduce the selction problem. \pause
  \begin{block}{The Selection Problem}
    \begin{itemize}
      \item Input: An array $A$ of $n$ numbers and an index $k$ with
      $0 \leq k < n$.
      \item Output: The element at index $k$ in the sorted array reordered
      from $A$. \pause
    \end{itemize}
  \end{block}
  Note that the output is the $(k+1)$-th smallest number of $A$, which is the
  $(k+1)$-th \emph{order statistic} of $A$. \pause
  \begin{itemize}
    \item It is useful in many cases, e.g., the $\lfloor(n-1)/2\rfloor$-th
    order statistic is the \emph{lower median}.
  \end{itemize}
\end{frame}

\section{Attempts}
\subsection{\#1}
\begin{frame}[fragile]{The First Attempt}
  One can sort the array and then output the element at index $k$.
  \begin{block}{}
    \scriptsize
    \begin{lstlisting}[gobble=4]
    def sort_and_select(A, k):
        B = sorted(A)
        return B[k]
    \end{lstlisting}
    \pause
  \end{block}
  What is its time complexity? \pause
  \begin{itemize}
    \item One needs $\Theta(n \log n)$ time to sort an array with length $n$.
    \pause
    \item Indexing takes time $\Theta(1)$. \pause
  \end{itemize}
  Thus, this algorithm runs in $\Theta(n \log n)$ time.
\end{frame}

\subsection{\#2}
\begin{frame}[fragile]{The Second Attempt}
  If $k$ is small, we can find only the $(k+1)$ smallest numbers.
  \begin{block}{}
    \scriptsize
    \begin{lstlisting}[gobble=4]
    def partial_sort_and_select(A, k):
        B = list(A)
        for i in range(k + 1):
            for j in range(i + 1, len(A)):
                if B[j] < B[i]:
                    B[i], B[j] = B[j], B[i]
        return B[k]
    \end{lstlisting}
    \pause
  \end{block}
  What is its time complexity? \pause
  \begin{itemize}
    \item Swaping two elements takes $\Theta(1)$ time. \pause
    \item The number of the inner iterations is
    \small
    \begin{equation*}
      (n-1) + (n-2) + \cdots + (n-k-1) = \left(n-\frac{k+2}{2}\right)(k+1).
    \end{equation*}
    \pause
  \end{itemize}
  \vspace{-1em}
  Thus, this algorithm runs in $\Theta(nk)$ time.
\end{frame}

\subsection{Comparison}
\begin{frame}{Comparison of Two Attempts}
  Let us compare the two algorithms. \pause
  \begin{itemize}
    \item The first attempt runs in $\Theta(n \log n)$ time. \pause
    \item The second attempt runs in $\Theta(nk)$ time. \pause
  \end{itemize}
  Which one is faster? \pause
  It depends on the size of $k$. \pause
  \begin{itemize}
    \item If $k$ is large (i.e., $k = \Omega(\log n)$), we can choose the first
    method. \pause
    \item If $k$ is small (i.e., $k = O(\log n)$), we can choose the second
    method. \pause
  \end{itemize}
  Thus, we have an algorithm that runs in $\Theta(\min\{n \log n, nk\})$ time.
\end{frame}

\section{Quick Select}
\subsection{Introduction}
\begin{frame}{Quicker, Quicker!}
  However, there exists faster algorithms that solve the selction problem.
  \pause
  \begin{itemize}
    \item A deterministic algorithm, called the \emph{median of medians}
    algorithm, solves the problem in $O(n)$ time. \pause
    \item A randomized algorithm, called the \emph{quick select} algorithm,
    solves the problem in expected $O(n)$ time. \pause
  \end{itemize}
  We are going to introduce the latter one, and thus we need to deal with
  \emph{randomization} first.
\end{frame}

\subsection{Randomization}
\begin{frame}[fragile]{Randomization}
  We use the package \lstinline{random} to generate pseudo-random numbers.
  \pause
  \begin{block}{}
    \scriptsize
    \begin{lstlisting}[gobble=4]
    >>> import random
    \end{lstlisting}
    \pause
  \end{block}

  \begin{block}{}
    \scriptsize
    \begin{lstlisting}[gobble=4]
    >>> random.random()    # generate number in [0.0, 1.0)
    0.5476045155149599
    \end{lstlisting}
    \pause
    \begin{lstlisting}[gobble=4]
    >>> random.random()
    0.630346861646684
    \end{lstlisting}
    \pause
  \end{block}

  \begin{block}{}
    \scriptsize
    \begin{lstlisting}[gobble=4]
    >>> random.randrange(5)    # generate integer in [0, 5)
    2
    \end{lstlisting}
    \pause
    \begin{lstlisting}[gobble=4]
    >>> random.randrange(5)
    4
    \end{lstlisting}
    \pause
    \begin{lstlisting}[gobble=4]
    >>> random.randrange(5)
    3
    \end{lstlisting}
    \pause
  \end{block}
  See the
  \href{https://docs.python.org/3/library/random.html}
       {offcial documentation page}
  for more examples.
\end{frame}

\subsection{Partition}
\begin{frame}[fragile]{Partition}
  \setlength{\topsep}{0pt}
  \setlength{\partopsep}{0pt}
  Now we introduce the partition algorithm, which is really useful. \pause
  \begin{itemize}
    \item First, we choose the last element as \emph{pivot}. \pause
    \item Then we reorder the array such that elements less than pivot come
    before the pivot, while elements greater than pivot come after the pivot.
    \pause
  \end{itemize}
  \begin{block}{}
    \scriptsize
    \begin{lstlisting}[gobble=4]
    def partition(A, l, r):
        i = l
        for j in range(l, r - 1):
            if A[j] < A[r - 1]:
                A[i], A[j] = A[j], A[i]
                i += 1
        A[i], A[r - 1] = A[r - 1], A[i]
        return i
    \end{lstlisting}
  \end{block}
\end{frame}

\begin{frame}{Partition (cont.)}
  \begin{center}
    \begin{tikzpicture}[scale=0.65,every node/.style={scale=0.65}]
      \def\l{7}
      \begin{scope}[shift={(0, 0)}]
        \def\j{1}
        \def\k{1}
        \node at (-1, 0) {Step 1};
        \foreach \x [count=\i] in {3, 1, 4, 1, 5, 9, 2} {
          \draw[
            shift={(\i, 0)},
            {\ifnum \i<\j CustomBlue\else
            \ifnum \i<\k CustomRed\else
            \ifnum \i<\l black!30\else
            CustomGreen\fi\fi\fi},
            fill={\ifnum \i<\j CustomBlue!10\else
                  \ifnum \i<\k CustomRed!10\else
                  \ifnum \i<\l white\else
                  CustomGreen!10\fi\fi\fi},
          ] (-0.45, -0.45) rectangle node {\x} (0.45, 0.45);
        }
      \end{scope}
      \pause
      \begin{scope}[shift={(0, -1.2)}]
        \def\j{1}
        \def\k{2}
        \node at (-1, 0) {Step 2};
        \foreach \x [count=\i] in {3, 1, 4, 1, 5, 9, 2} {
          \draw[
            shift={(\i, 0)},
            {\ifnum \i<\j CustomBlue\else
            \ifnum \i<\k CustomRed\else
            \ifnum \i<\l black!30\else
            CustomGreen\fi\fi\fi},
            fill={\ifnum \i<\j CustomBlue!10\else
                  \ifnum \i<\k CustomRed!10\else
                  \ifnum \i<\l white\else
                  CustomGreen!10\fi\fi\fi},
          ] (-0.45, -0.45) rectangle node {\x} (0.45, 0.45);
        }
      \end{scope}
      \pause
      \begin{scope}[shift={(0, -2.4)}]
        \def\j{2}
        \def\k{3}
        \node at (-1, 0) {Step 3};
        \foreach \x [count=\i] in {1, 3, 4, 1, 5, 9, 2} {
          \draw[
            shift={(\i, 0)},
            {\ifnum \i<\j CustomBlue\else
            \ifnum \i<\k CustomRed\else
            \ifnum \i<\l black!30\else
            CustomGreen\fi\fi\fi},
            fill={\ifnum \i<\j CustomBlue!10\else
                  \ifnum \i<\k CustomRed!10\else
                  \ifnum \i<\l white\else
                  CustomGreen!10\fi\fi\fi},
          ] (-0.45, -0.45) rectangle node {\x} (0.45, 0.45);
        }
      \end{scope}
      \pause
      \begin{scope}[shift={(0, -3.6)}]
        \def\j{2}
        \def\k{4}
        \node at (-1, 0) {Step 4};
        \foreach \x [count=\i] in {1, 3, 4, 1, 5, 9, 2} {
          \draw[
            shift={(\i, 0)},
            {\ifnum \i<\j CustomBlue\else
            \ifnum \i<\k CustomRed\else
            \ifnum \i<\l black!30\else
            CustomGreen\fi\fi\fi},
            fill={\ifnum \i<\j CustomBlue!10\else
                  \ifnum \i<\k CustomRed!10\else
                  \ifnum \i<\l white\else
                  CustomGreen!10\fi\fi\fi},
          ] (-0.45, -0.45) rectangle node {\x} (0.45, 0.45);
        }
      \end{scope}
      \pause
      \begin{scope}[shift={(0, -4.8)}]
        \def\j{3}
        \def\k{5}
        \node at (-1, 0) {Step 5};
        \foreach \x [count=\i] in {1, 1, 4, 3, 5, 9, 2} {
          \draw[
            shift={(\i, 0)},
            {\ifnum \i<\j CustomBlue\else
            \ifnum \i<\k CustomRed\else
            \ifnum \i<\l black!30\else
            CustomGreen\fi\fi\fi},
            fill={\ifnum \i<\j CustomBlue!10\else
                  \ifnum \i<\k CustomRed!10\else
                  \ifnum \i<\l white\else
                  CustomGreen!10\fi\fi\fi},
          ] (-0.45, -0.45) rectangle node {\x} (0.45, 0.45);
        }
      \end{scope}
      \pause
      \begin{scope}[shift={(0, -6)}]
        \def\j{3}
        \def\k{6}
        \node at (-1, 0) {Step 6};
        \foreach \x [count=\i] in {1, 1, 4, 3, 5, 9, 2} {
          \draw[
            shift={(\i, 0)},
            {\ifnum \i<\j CustomBlue\else
            \ifnum \i<\k CustomRed\else
            \ifnum \i<\l black!30\else
            CustomGreen\fi\fi\fi},
            fill={\ifnum \i<\j CustomBlue!10\else
                  \ifnum \i<\k CustomRed!10\else
                  \ifnum \i<\l white\else
                  CustomGreen!10\fi\fi\fi},
          ] (-0.45, -0.45) rectangle node {\x} (0.45, 0.45);
        }
      \end{scope}
      \pause
      \begin{scope}[shift={(0, -7.2)}]
        \def\j{3}
        \def\k{7}
        \node at (-1, 0) {Step 7};
        \foreach \x [count=\i] in {1, 1, 4, 3, 5, 9, 2} {
          \draw[
            shift={(\i, 0)},
            {\ifnum \i<\j CustomBlue\else
            \ifnum \i<\k CustomRed\else
            \ifnum \i<\l black!30\else
            CustomGreen\fi\fi\fi},
            fill={\ifnum \i<\j CustomBlue!10\else
                  \ifnum \i<\k CustomRed!10\else
                  \ifnum \i<\l white\else
                  CustomGreen!10\fi\fi\fi},
          ] (-0.45, -0.45) rectangle node {\x} (0.45, 0.45);
        }
      \end{scope}
      \pause
      \begin{scope}[shift={(0, -8.4)}]
        \def\j{3}
        \def\k{4}
        \node at (-1, 0) {Step 8};
        \foreach \x [count=\i] in {1, 1, 2, 3, 5, 9, 4} {
          \draw[
            shift={(\i, 0)},
            {\ifnum \i<\j CustomBlue\else
            \ifnum \i<\k CustomGreen\else
            \ifnum \i<\l CustomRed\else
            CustomRed\fi\fi\fi},
            fill={\ifnum \i<\j CustomBlue!10\else
                  \ifnum \i<\k CustomGreen!10\else
                  \ifnum \i<\l CustomRed!10\else
                  CustomRed!10\fi\fi\fi},
          ] (-0.45, -0.45) rectangle node {\x} (0.45, 0.45);
        }
      \end{scope}
      \onslide<1->
    \end{tikzpicture}
  \end{center}
\end{frame}

\subsection{Quick Select}
\begin{frame}[fragile]{Quick Select}
  With the partition procedure, one can select the any order statistic really
  quickly!
  \begin{block}{}
    \scriptsize
    \begin{lstlisting}[gobble=4]
    def quick_select(A, l, r, k):
        i = partition(A, l, r)
        if k == i:
            return A[k]
        elif k > i:
            return quick_select(A, i + 1, r, k)
        else:
            return quick_select(A, l, i, k)
    \end{lstlisting}
  \end{block}
\end{frame}

\subsection{Analysis}
\begin{frame}{Analysis of Quick Select}
  Let us analyze the time complexity of quick select. \pause
  \begin{itemize}
    \item The partition procedure runs in $\Theta(n)$ time when the array has
    $n$ elements. \pause
    \item If the partition is balanced, then we have
    \begin{equation*}
      T(n) = T(n/2) + \Theta(n)
    \end{equation*}
    for $n \geq 2$, implying $T(n) = \Theta(n)$. \pause
    \item However, if the partition is unbalanced, then we may have
    \begin{equation*}
      T(n) = T(n - 1) + \Theta(n)
    \end{equation*}
    for $n \geq 2$, implying $T(n) = \Theta(n^2)$.
  \end{itemize}
\end{frame}

\begin{frame}{Analysis of Quick Select (cont.)}
  If we'd like to make the partition balanced, we can use some tricks. \pause
  \begin{itemize}
    \item Instead of choosing the last element as pivot, we can choose a random
    element as pivot. \pause
    \item With randomization, the partition is somehow balanced in most cases,
    and it leads to a expected linear time complexity.
  \end{itemize}
\end{frame}

\subsection{Conclusion}
\begin{frame}{Conclusion}
  The time complexity of the selction problem is $\Theta(n)$.
  \begin{block}{The Selection Problem}
    \begin{itemize}
      \item Input: An array $A$ of $n$ numbers and an index $k$ with
      $0 \leq k < n$.
      \item Output: The element at index $k$ in the sorted array reordered
      from $A$. \pause
    \end{itemize}
  \end{block}
  The comparison among the algorithms that solves the selction problem is as
  follows.
  \begin{center}
    \def\arraystretch{1.5}
    \begin{tabular}{|c|c|}
      \hline
      Algorithm & Worst-case Time Complexity \\
      \hline
      Sorting (Merge Sort) & $\Theta(n \log n)$ \\
      Partial Selection Sort & $\Theta(nk)$ \\
      Quick Select & $\Theta(n)$ (expected) \\
      Median of Medians & $\Theta(n)$ \\
      \hline
    \end{tabular}
  \end{center}
\end{frame}

\section{Exercises}
\subsection{\#1}
\begin{frame}{Exercise \#1}
  If $T(n)$ satisfies
  \begin{equation*}
    T(n) =
    \begin{cases}
      O(1), &\text{if $n \leq 1$} \\
      T(n - 2) + \Theta(n), &\text{otherwise},
    \end{cases}
  \end{equation*}
  what is the exact complexity of $T(n)$? \pause
  \begin{block}{Solution}
    We have $T(n) = \Theta(n^2)$.
  \end{block}
\end{frame}

\subsection{\#2}
\begin{frame}{Exercise \#2}
  If $T(n)$ satisfies
  \begin{equation*}
    T(n) =
    \begin{cases}
      O(1), &\text{if $n \leq 1$} \\
      T(9n/10) + \Theta(n), &\text{otherwise},
    \end{cases}
  \end{equation*}
  what is the exact complexity of $T(n)$? \pause
  \begin{block}{Solution}
    Use the master theorem, and we have $T(n) = \Theta(n)$.
  \end{block}
\end{frame}

\subsection{\#3}
\begin{frame}[fragile]{Exercise \#3}
  In fact, there is an algorithm called quick sort, that also uses the
  partition procedure. \pause
  \begin{block}{}
    \scriptsize
    \begin{lstlisting}[gobble=4]
    def quick_sort(A, l, r):
        if r - l <= 1:
            return
        i = partition(A, l, r)
        quick_sort(A, l, i)
        quick_sort(A, i + 1, r)
    \end{lstlisting}
    \pause
  \end{block}
  If you are really unlucky such that the partition is unbalanced in each
  iteration, how much time will it takes to run quick sort? \pause
  \begin{itemize}
    \item You can assume that the pivot is always the largest element. \pause
  \end{itemize}
  \begin{block}{Solution}
    It takes $\Theta(n^2)$ time. \\
    (However, quick sort can run in expected $\Theta(n \log n)$ time if
    randomization is used.)
  \end{block}
\end{frame}

\end{document}