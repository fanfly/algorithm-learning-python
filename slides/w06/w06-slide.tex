\documentclass{beamer}
\usepackage{../mysty}

\title{Week 06: Tasks}
\author{FanFly}
\date{April 12, 2020}

\begin{document}

\begin{frame}
  \titlepage
\end{frame}

\section{Task Selction Problem}
\subsection{Definition}
\begin{frame}{The Task Selction Problem}
  Today we are going to introduce another combinatorial optimzation problem,
  which is called the \emph{task selection problem}. \pause
  \begin{block}{Task Selection Problem}
    \begin{itemize}
      \item Input: A collection of $n$ tasks, each with a start time $x_i$
      and an end time $y_i$.
      \item Output: A maximum size subset of mutually compatible tasks.
    \end{itemize}
    \pause
  \end{block}
  We will use $T_i = [x_i, y_i)$ to denote the $i$th task. \pause \\[1em]
  Furthermore, two tasks $T_i$ and $T_j$ are compatible if
  \begin{equation*}
    [x_i, y_i) \cap [x_j, y_j) = \varnothing.
  \end{equation*}
\end{frame}

\begin{frame}{Compatible Tasks}
  Given a set of tasks $\{T_1, \dots, T_n\}$, how can we check if they are
  mutually compatible? \pause
  \begin{itemize}
    \item Check if all pairs of tasks are disjoint.
    It takes $\Theta(n^2)$ time. \pause
    \item Sort the tasks based on start time, and then check if all consecutive
    tasks are disjoint.
    It takes $\Theta(n)$ time to check the tasks. \pause
  \end{itemize}
  \vspace{.5em}
  Sorting saves lots of time! \pause \\[.5em]
  Thus, it must be a good idea to sort the tasks first.
\end{frame}

\subsection{Brute-Force Method}
\begin{frame}{A Brute-force Approach}
  Now supposed that the tasks are sorted based on start time. \pause
  \begin{itemize}
    \item For each possible subset of tasks, check if it contains tasks that
    are not disjoint. \pause
    \item There are $\Theta(2^n)$ possible subsets. \pause
    \item It takes $O(n)$ time for each subset. \pause
    \item Thus, we can solve the problem in $O(2^n n)$ time.
  \end{itemize}
\end{frame}

\begin{frame}[fragile]{A Brute-force Approach}
  By the way, how to enumerate all subsets of a set? \pause
  \begin{block}{}
    \scriptsize
    \begin{lstlisting}[gobble=4]
    def subsets(n):
        """Returns all subsets of [0, 1, ..., n-1]."""
        subsets = []
        for m in range(2 ** n):
            subset = []
            for i in range(n):
                if (m // (2 ** i)) % 2 == 1:
                    subset.append(i)
            subsets.append(subset)
        return subsets
    \end{lstlisting}
  \end{block}
  \pause
  \begin{block}{}
    \scriptsize
    \begin{lstlisting}[gobble=4]
    >>> subsets(3)
    \end{lstlisting}
    \pause
    \begin{lstlisting}[gobble=4]
    [[], [0], [1], [0, 1], [2], [0, 2], [1, 2], [0, 1, 2]]
    \end{lstlisting}
  \end{block}
\end{frame}

\subsection{Greedy Method}
\begin{frame}{A Simple Observation}
  The following theorem is a simple result, but it can help us develop a faster
  algorithm. \pause
  \begin{block}{Theorem}
    Let $S = \{T_1, \dots, T_n\}$ be a collection of tasks. \pause
    \begin{itemize}
      \item If $T_i \subseteq T_j$, then there must be a maximum-size subset of
      mutually compatible tasks that does not contain $T_j$. \pause
      \item The task with earliest finish time must be included in some
      maximum-size subset of mutually compatible tasks.
    \end{itemize}
  \end{block}
\end{frame}

\begin{frame}{Greedy Algorithm}
  This theorem leads us to find the following algorithm. \pause \\[1em]
  For each step, let $S = \{T_1, \dots, T_n\}$ be the set of tasks that we are
  currently dealing with, where $T_1, \dots, T_n$ are sorted based on start
  time. \pause
  \begin{itemize}
    \item Note that $T_1 = [x_1, y_1)$ is the task with earliest start time.
    \pause
    \item If $y_1 \geq y_2$, then we discard $T_1$. \pause
    \item If $y_1 < y_2$, then we choose $T_1$ and discard the tasks
    overlapping with $T_1$. \pause
  \end{itemize}
  Since we always make the choice that looks best at the moment, the algorithm
  is usually called a \emph{greedy algorithm}.
\end{frame}

\section{Exercises}
\subsection{\#1}
\begin{frame}{Exercise \#1}
  \begin{block}{Task Selection Problem}
    \begin{itemize}
      \item Input: A collection of $n$ tasks, each with a start time $x_i$
      and an end time $y_i$.
      \item Output: A maximum size subset of mutually compatible tasks.
    \end{itemize}
  \end{block}
  What is the overall time complexity of the greedy algorithm that solves the
  task selection problem?
  \begin{multicols}{4}
    \begin{itemize}
      \item $\Theta(n)$
      \item $\Theta(n \log n)$
      \item $\Theta(n^2)$
      \item $\Theta(2^n)$ \pause
    \end{itemize}
  \end{multicols}
  \begin{block}{Solution}
    It takes $\Theta(n \log n)$ time to sort the tasks based on start time.
    Then it takes $\Theta(n)$ to choose the compatible tasks.
    Thus, the overall time complexity is $\Theta(n \log n)$.
  \end{block}
\end{frame}

\subsection{\#2}
\begin{frame}[fragile]{Exercise \#2}
  Consider the following code, which is slightly different from the previous
  version.
  \begin{block}{}
    \scriptsize
    \begin{lstlisting}[gobble=4]
    def subsets(n):
        """Returns all subsets of [0, 1, ..., n-1]."""
        subsets = {}  # use dict instead of list
        for m in range(2 ** n):
            subset = []
            for i in range(n):
                if (m // (2 ** i)) % 2 == 1:
                    subset.append(i)
            subsets[m] = subset
        return subsets
    \end{lstlisting}
  \end{block}
  \pause
  Then we have the following result.
  \begin{block}{}
    \scriptsize
    \begin{lstlisting}[gobble=4]
    >>> subsets(2)
    {0: [], 1: [0], 2: [1], 3: [0, 1]}
    \end{lstlisting}
  \end{block}
\end{frame}

\begin{frame}[fragile]{Exercise \#2 (cont.)}
  Now we assume that the following Python code is executed.
  \begin{block}{}
    \scriptsize
    \begin{lstlisting}[gobble=4]
    n = 5
    P = subsets(n)
    \end{lstlisting}
  \end{block}
  Please find the value of \lstinline{P[18]}. \pause
  \begin{block}{Solution}
    Since $18 = 2^1 + 2^4$, the value of \lstinline{P[18]} should be
    \lstinline{[1, 4]}. \\
    (That is, 18 can be represented as 10010 in binary.)
  \end{block}
\end{frame}

\subsection{\#3}
\begin{frame}{Exercise \#3}
  Consider the task selection problem for the following set of tasks.
  \begin{center}
    \begin{tabular}{c|ccccccccccc}
      Task & $T_1$ & $T_2$ & $T_3$ & $T_4$ & $T_5$ & $T_6$ & $T_7$ & $T_8$ &
             $T_9$ & $T_{10}$ & $T_{11}$ \\
      \hline
      Start & 0 & 1 & 2 & 3 & 3 & 5 & 5 & 6 & 8 & 8 & 12 \\
      End & 6 & 4 & 14 & 5 & 9 & 7 & 9 & 10 & 11 & 12 & 16 \\
    \end{tabular}
  \end{center}
  What is the maximum size of subset of mutually compatible tasks? \pause
  \begin{block}{Solution}
    There can be at most 4 mutually compatible tasks, e.g.,
    $\{T_2, T_6, T_9, T_{11}\}$.
  \end{block}
\end{frame}

\end{document}