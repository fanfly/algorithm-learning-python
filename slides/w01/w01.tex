\documentclass{beamer}
\usetheme{CambridgeUS}

\definecolor{CustomBlue}{rgb}{0.2588, 0.5216, 0.9569}
\definecolor{CustomRed}{rgb}{0.8588, 0.2667, 0.2157}
\definecolor{CustomYellow}{rgb}{0.9569, 0.6275, 0}
\definecolor{CustomGreen}{rgb}{0.0588, 0.6157, 0.3451}
\setbeamercolor{palette primary}{bg=CustomBlue!70,fg=black}
\setbeamercolor{palette secondary}{bg=CustomYellow!75,fg=black}
\setbeamercolor{palette tertiary}{bg=CustomRed!70,fg=black}
\setbeamercolor{structure}{fg=CustomGreen}
\setbeamercolor{title}{fg=CustomGreen!60!black,bg=CustomGreen!5}
\setbeamercolor{frametitle}{fg=CustomGreen!60!black,bg=CustomGreen!5}
\setbeamercolor{block title}{fg=white,bg=CustomGreen}
\setbeamercolor{block body}{bg=CustomGreen!10}

\usepackage{listings}
\lstset{
  basicstyle=\ttfamily,
  keywordstyle=\color{blue},
  stringstyle=\color{red},
  commentstyle=\color{gray!70!CustomGreen},
  language=Python,
  aboveskip=0pt,
  belowskip=0pt,
  showstringspaces=false,
}

\usepackage{multicol}
% 0.5 times of original value
\setlength{\multicolsep}{6.0pt plus 2.0pt minus 1.5pt}

% use bold fonts to emphasize
\DeclareTextFontCommand{\emph}{\boldmath\bfseries}

\title{Week 01: Introduction to Algorithms}
\author{FanFly}
\date{March 8, 2020}

\begin{document}

\begin{frame}
  \titlepage
\end{frame}

\section{Computational Problems and Algorithms}
\subsection{Computational Problems}
\begin{frame}{Computational Problems}
  What is a computational problem? \pause
  \begin{itemize}
    \item A \emph{computational problem} is a relation between inputs and
    outputs.
  \end{itemize}
\end{frame}

\begin{frame}{The Primality Test Problem}
  Here is an example. \pause
  \begin{block}{The Primality Test Problem}
    \pause
    \begin{itemize}
      \item Input: A positive integer $m \geq 2$. \pause
      \item Output: True if $m$ is prime, false if $m$ is composite. \pause
    \end{itemize}
  \end{block}
  Let's try to solve it for some instances by hand. \pause
  \begin{itemize}
    \item Is 11 a prime number? \pause
    Yes, it is. \pause
    \item Is 111 a prime number? \pause
    No, it isn't since $111 = 3 \times 37$. \pause
    \item Is 1111 a prime number? \pause
    No, it isn't since $1111 = 11 \times 101$. \pause
    \item Is 11111 a prime number? \pause
    No, it isn't since $11111 = 41 \times 271$. \pause
    \item Is 1111111111111111111 a prime number? \pause
    Yes, it is.
  \end{itemize}
\end{frame}

\subsection{Algorithms}
\begin{frame}[fragile]{Algorithms}
  It's too difficult to test if an integer is prime by hand. \\
  Thus, people write programs to solve it. \pause
  \begin{block}{}
    \scriptsize
    \begin{lstlisting}[gobble=4]
    def is_prime(m):
        k = 2
        while k < m:
            if m % k == 0:
                return False
            k += 1
        return True
    \end{lstlisting}
    \pause
  \end{block}
  This function (in Python) produces a correct output for each possible
  input of the problem. \pause
  \begin{itemize}
    \item If a function produces a correct output for each possible
    input of a problem, we say that it \emph{solves} the problem.
    \item Formally, we call such a function an \emph{algorithm}.
  \end{itemize}
\end{frame}

\subsection{Measurement}
\begin{frame}{Hardness of a Problem}
  In most cases, we would like to measure the ``hardness'' of a problem. \pause
  \begin{itemize}
    \item Assume that there is a program (i.e., an algorithm) that solves the
    problem. \pause
    \item We can measure how long it takes to execute the program. \pause
    \item We can measure how much memory it uses to execute the program. \pause
  \end{itemize}
  However, the measurement may not be consistent due to some reasons. \pause
  \begin{itemize}
    \item The input may change. \pause
    \item The programming language may change. \pause
    \item The power of the computer (on which the program runs) may change.
  \end{itemize}
\end{frame}

\begin{frame}{Computation Models}
  In order to make the measurement consistent in any situation, we have to
  specify a computation model. \pause \\[.5em]
  For example, we can define the following operations as \emph{primitive} ones,
  each consumes one unit of time.
  \pause
  \begin{itemize}
    \item Assigning a value to a variable. \pause
    \item Performing an arithmetic or logical operation. \pause
    \item Indexing into an array (i.e., a \lstinline{list} in Python). \pause
    \item Performing a jump due to branch, loop or function call. \pause
  \end{itemize}
  Then the executation time can be measured regardless of the environment.
\end{frame}

\section{Asymptotic Analysis}
\subsection{Time Complexity}
\begin{frame}[fragile]{Analysis of Time Complexity}
  Let us consider this program again.
  \begin{block}{}
    \scriptsize
    \begin{lstlisting}[gobble=4]
    def is_prime(m):
        k = 2
        while k < m:
            if m % k == 0:
                return False
            k += 1
        return True
    \end{lstlisting}
    \pause
  \end{block}
  Given the input $m$, it can be shown that the statements inside the
  \lstinline{while} loop is executed at most $m - 2$ times. \pause
  \begin{itemize}
    \item Note that the \lstinline{return} statement inside the
    \lstinline{while} loop is not executed if $m$ is prime. \pause
    \item In this case, the running time is ``approximately'' proportional
    to $m$. \pause
    \item If the running time of a program is ``approximately'' proportional to
    $m$, we say that the \emph{time complexity} of this program is $O(m)$.
  \end{itemize}
\end{frame}

\subsection{The Big-O Notation}
\begin{frame}{The Big-O Notation}
  What does $O(\cdot)$ mean?
  We state the definition formally. \pause
  \begin{block}{Definition (Big-O Notation)}
    Let $f: \mathbb{N} \to \mathbb{R}$ and $g: \mathbb{N} \to \mathbb{R}$ be
    functions.
    If there exists a constant $c > 0$ and an integer $N$ such that
    \begin{equation*}
      0 \leq f(n) \leq cg(n)
    \end{equation*}
    for each $n \geq N$, then we write $f(n) = O(g(n))$.
  \end{block}
\end{frame}

\begin{frame}{Examples of Big-O Notation}
  Following are some examples. \pause
  \begin{itemize}
    \item Is $1365n = O(n)$ true? \pause
    Yes, it is since $1365n \leq cn$ with $c = 1365$. \pause
    \item Is $n^2 = O(n)$ true? \pause
    No, it isn't since $n^2 > cn$ if $n > \lceil c \rceil$.
  \end{itemize}
\end{frame}

\subsection{Input Size}
\begin{frame}{Input Size}
  Now we have an algorithm that solves the primality test problem in $O(m)$
  time.
  \begin{block}{The Primality Test Problem}
    \begin{itemize}
      \item Input: An $n$-bit positive integer $m \geq 2$.
      \item Output: True if $m$ is prime, false if $m$ is composite. \pause
    \end{itemize}
  \end{block}
  Can we use $n$ (instead of $m$) to represent the time complexity of the
  algorithm? \pause
  \begin{itemize}
    \item An $n$-bit binary number can represent up to $2^n - 1$. \pause
    \item One should use at least $n = \lceil\log_2 (m + 1)\rceil$ bits
    to represent a positive integer $m$. \pause
    \item Thus the time complexity of the algorithm is $O(2^n)$.
  \end{itemize}
\end{frame}

\subsection{Comparison between Algorithms}
\begin{frame}{Comparison between Algorithms}
  Suppose that we have two algorithms that solve the same problem. \pause
  \begin{itemize}
    \item Algorithm A runs in $O(n^2)$ time. \pause
    \item Algorithm B runs in $O(n)$ time. \pause
  \end{itemize}
  Can we thus say that Algorithm B is better than Algorithm A? \pause \\[.5em]
  No! Note that we also have the following results. \pause
  \begin{itemize}
    \item Algorithm A runs in $O(n^3)$ time. \pause
    \item Algorithm B runs in $O(n^4)$ time. \pause
  \end{itemize}
  Thus, using big-O notation is not enough to compare the efficiency of
  different algorithms.
\end{frame}

\begin{frame}{The Big-Omega and Big-Theta Notations}
  Let us introduce new notations, $\Omega(\cdot)$ and $\Theta(\cdot)$. \pause
  \begin{block}{Definition (Big-Omega and Big-Theta Notations)}
    Let $f: \mathbb{N} \to \mathbb{R}$ and $g: \mathbb{N} \to \mathbb{R}$ be
    functions. \pause
    \begin{itemize}
      \item We write $f(n) = \Omega(g(n))$ if $g(n) = O(f(n))$. \pause
      \item We write $f(n) = \Theta(g(n))$ if $f(n) = O(g(n))$ and
      $g(n) = O(f(n))$. \pause
    \end{itemize}
  \end{block}
  Suppose again that we have two algorithms that solve the same problem. \pause
  \begin{itemize}
    \item Algorithm A runs in $\Omega(n^2)$ time. \pause
    \item Algorithm B runs in $O(n)$ time. \pause
  \end{itemize}
  Now we can say that Algorithm B is better than Algorithm A (in time
  complexity).
\end{frame}

\section{Exercises}
\subsection{\#1}
\begin{frame}[fragile]{Exercise \#1}
  We know that there is an $O(2^n)$-time algorithm that solves the primality
  test problem for $n$-bit inputs as follows.
  \begin{block}{}
    \scriptsize
    \begin{lstlisting}[gobble=4]
    def is_prime(m):
        k = 2
        while k < m:
            if m % k == 0:
                return False
            k += 1
        return True
    \end{lstlisting}
    \pause
  \end{block}
  In fact, it runs in time $\Theta(2^n)$ if we consider the worst-case
  complexity.
\end{frame}

\begin{frame}[fragile]{Exercise \#1 (cont.)}
  Now we have an improvement on this algorithm.
  \begin{block}{}
    \scriptsize
    \begin{lstlisting}[gobble=4]
    def is_prime(m):
        k = 2
        # Only test k <= sqrt(m)
        while k * k <= m:
            if m % k == 0:
                return False
            k += 1
        return True
    \end{lstlisting}
    \pause
  \end{block}
  What is the worst-case time complexity of this algorithm?
  \begin{multicols}{4}
    \begin{itemize}
      \item $\Theta(n)$
      \item $\Theta(2^{\sqrt{n}})$
      \item $\Theta(2^{n/2})$
      \item $\Theta(2^n)$ \pause
    \end{itemize}
  \end{multicols}
  \begin{block}{Solution}
    The worst-case time complexity is $\Theta(\sqrt{2^n}) = \Theta(2^{n/2})$.
  \end{block}
\end{frame}

\subsection{\#2}
\begin{frame}{Exercise \#2}
  Let $f: \mathbb{N} \to \mathbb{R}$ be a function with
  \begin{equation*}
    f(n) = \sum_{k=1}^n \frac{1}{k}
    = \frac{1}{1} + \frac{1}{2} + \cdots + \frac{1}{n}
  \end{equation*}
  for any positive integer $n$. \pause \\[.5em]
  Find an elementary function $g: \mathbb{N} \to \mathbb{R}$ such that
  $f(n) = \Theta(g(n))$. \pause
  \begin{block}{Solution}
    We have $f(n) = \Theta(\ln n)$ since
    \begin{equation*}
      \int_1^{n+1} \frac{1}{t}\,dt
      \quad \leq \quad f(n)
      \quad \leq \quad 1 + \int_1^n \frac{1}{t}\,dt.
    \end{equation*}
  \end{block}
\end{frame}

\subsection{\#3}
\begin{frame}[fragile]{Exercise \#3}
  Consider the following function.
  \begin{block}{}
    \scriptsize
    \begin{lstlisting}[gobble=4]
    def hello(n):
        if n == 0:
            return
        elif n == 1:
            print("Hello!")
        else:
            hello(n - 2)
            hello(n - 1)
    \end{lstlisting}
    \pause
  \end{block}
  Let $H(n)$ be the number of lines of \lstinline{"Hello!"} printed by this
  function, given the input $n$. \pause \\[.5em]
  Then we have
  \begin{equation*}
    H(n) =
    \begin{cases}
      0, & \text{if $n = 0$} \\
      1, & \text{if $n = 1$} \\
      H(n-2) + H(n-1), & \text{otherwise}.
    \end{cases}
  \end{equation*}
\end{frame}

\begin{frame}{Exercise \#3 (cont.)}
  Which of the following is true?
  \begin{itemize}
    \item $H(n) = O(n)$.
    \item $H(n) = \Omega(n)$ and $H(n) = O(n^2)$.
    \item $H(n) = \Omega(n^2)$ and $H(n) = O(2^n)$.
    \item $H(n) = \Omega(2^n)$. \pause
  \end{itemize}
  \begin{block}{Solution}
    Note that $H(n)$ is the $n$th Fibonacci number. \pause
    Thus we have
    \begin{equation*}
      H(n)
      = \frac{\left(\frac{1+\sqrt{5}}{2}\right)^n
        - \left(\frac{1-\sqrt{5}}{2}\right)^n}{\sqrt{5}}
      = \Theta\left(\left(\frac{1+\sqrt{5}}{2}\right)^n\right),
    \end{equation*}
    implying $H(n) = \Omega(n^2)$ and $H(n) = O(2^n)$.
  \end{block}
\end{frame}
\end{document}