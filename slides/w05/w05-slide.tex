\documentclass{beamer}
\usepackage{../mysty}

\title{Week 05: Combination}
\author{FanFly}
\date{April 5, 2020}

\begin{document}

\begin{frame}
  \titlepage
\end{frame}

\section{Knapsack Problem}
\subsection{Definition}
\begin{frame}{Knapsack Problem}
  Today we are going to introduce a combinatorial optimzation problem, which is
  called the \emph{knapsack problem}. \pause
  \begin{block}{Knapsack Problem}
    \begin{itemize}
      \item Input:
      \begin{itemize}
        \item A set of $n$ items, each with a weight $w_i$ and a value
        $v_i$.
        \item A knapsack with weight capacity $C$. \pause
      \end{itemize}
      \item Output: The most valuable combination of items that fits in the
      knapsack.
    \end{itemize}
    \pause
  \end{block}
  It is supposed that the weight capacity of knapsack and the weights and
  values of all items are positive integers.
\end{frame}

\begin{frame}{An Instance of Knapsack Problem}
  Following is an instance of knapsack problem. \pause
  \begin{center}
    \def\arraystretch{1.25}
    \begin{tabular}{|c|c|}
      \hline
      Weight Capacity & 10 \\
      \hline
    \end{tabular} \\[.5em]
    \begin{tabular}{|c|cc|}
      \hline
      Item & Weight & Value \\
      \hline
      A & 2 & 5 \\
      B & 3 & 7 \\
      C & 5 & 10 \\
      D & 8 & 19 \\
      \hline
    \end{tabular}
    \pause
  \end{center}
  It can be found that the most valuable combination is A and D, whose total
  value is 24.
\end{frame}

\subsection{Brute Force Algorithm}
\begin{frame}{An Exponential Time Algorithm}
  Since the number of all possible combinations is finite, we can perform a
  brute force algorithm. \pause
  \begin{itemize}
    \item There are $2^n$ different possible combinations of items. \pause
    \item For each combination, we need $O(n)$ time to compute the total weight
    and the total value. \pause
    \item Thus, there exists an $O(2^nn)$-time algorithm. \pause
  \end{itemize}
  \vspace{1em}
  The function $f(n) = 2^nn$ grows really fast as $n$ increases, so it can only
  be used for small $n$.
  \begin{center}
    \def\arraystretch{1.25}
    \begin{tabular}{|c|ccccc|}
      \hline
      $n$ & 1 & 5 & 10 & 50 & 100 \\
      \hline
      $f(n)$ & 2 & 160 & 10240 & $5.6 \times 10^{16}$ & $1.3 \times 10^{32}$ \\
      \hline
    \end{tabular}
  \end{center}
\end{frame}

\subsection{Dynamic Programming}
\begin{frame}{Dynamic Programming}
  If we want to solve the knapsack problem for big $n$, we can use dynamic
  programming! \pause \\[.5em]
  Let $V(W, S)$ be the maximum value when
  \begin{itemize}
    \item the weight capacity is $W$, and
    \item the items can only be chosen from the set $S$. \pause
  \end{itemize}
  \vspace{.5em}
  How to determine $V(W, S \cup \{k\})$? \pause
  \begin{itemize}
    \item If item $k$ is {\color{red}chosen}, then we can choose items from $S$
    with weight up to $W - w_k$. \pause
    \item If item $k$ is {\color{blue}not chosen}, then we can choose items
    from $S$ with weight up to $W$. \pause
    \item Thus, we have
    \begin{equation*}
      V(W, S \cup \{k\}) = \max\{
        {\color{red}v_k + V(W - w_k, S)},
        {\color{blue}V(W, S)}\}.
    \end{equation*}
  \end{itemize}
\end{frame}

\begin{frame}{Dynamic Programming (cont.)}
  \begin{center}
    \scriptsize
    \def\arraystretch{1.25}
    \begin{tabular}{|c|cccc|}
      \hline
      & A & B & C & D \\
      \hline
      Weight & 2 & 3 & 5 & 8 \\
      Value & 5 & 7 & 10 & 19 \\
      \hline
      \multicolumn{5}{c}{} \\[-1em]
      \hline
      & A & B & C & D \\
      \hline
      0 & \onslide<2->{0 & 0 & 0 & 0} \\
      1 & \onslide<3->{0 & 0 & 0 & 0} \\
      2 & \onslide<4->{5 & 5 & 5 & 5} \\
      3 & \onslide<5->{5 & 7 & 7 & 7} \\
      4 & \onslide<6->{5 & 7 & 7 & 7} \\
      5 & \onslide<7->{5 & 12 & 12 & 12} \\
      6 & \onslide<8->{5 & 12 & 12 & 12} \\
      7 & \onslide<9->{5 & 12 & 15 & 15} \\
      8 & \onslide<10->{5 & 12 & 17 & 19} \\
      9 & \onslide<11->{5 & 12 & 17 & 19} \\
      10 & \onslide<12->{5 & 12 & 17 & 24} \\
      \hline
    \end{tabular}
  \end{center}
\end{frame}

\begin{frame}{Dynamic Programming (cont.)}
  \begin{center}
    \scriptsize
    \def\arraystretch{1.25}
    \begin{tabular}{|c|cccc|}
      \hline
      & A & B & C & D \\
      \hline
      Weight & 2 & 3 & 5 & 8 \\
      Value & 5 & 7 & 10 & 19 \\
      \hline
      \multicolumn{5}{c}{} \\[-1em]
      \hline
      & \only<1>{A}\only<2>{\cellcolor{CustomBlue!40}A} & B & C &
      \only<1>{D}\only<2>{\cellcolor{CustomBlue!40}D} \\
      \hline
      0 & 0 & 0 & 0 & 0 \\
      1 & 0 & 0 & 0 & 0 \\
      2 & \cellcolor{CustomRed!40}5 & \cellcolor{CustomRed!40}5 &
      \cellcolor{CustomRed!40}5 & 5 \\
      3 & 5 & 7 & 7 & 7 \\
      4 & 5 & 7 & 7 & 7 \\
      5 & 5 & 12 & 12 & 12 \\
      6 & 5 & 12 & 12 & 12 \\
      7 & 5 & 12 & 15 & 15 \\
      8 & 5 & 12 & 17 & 19 \\
      9 & 5 & 12 & 17 & 19 \\
      10 & 5 & 12 & \cellcolor{CustomGreen!40}17 &
      \cellcolor{CustomRed!40}24 \\
      \hline
    \end{tabular}
  \end{center}
\end{frame}

\subsection{Conclusion}
\begin{frame}{Conclusion}
  \begin{block}{Knapsack Problem}
    \begin{itemize}
      \item Input:
      \begin{itemize}
        \item A set of $n$ items, each with a weight $w_i$ and a value
        $v_i$.
        \item A knapsack with weight capacity $C$.
      \end{itemize}
      \item Output: The most valuable combination of items that fits in the
      knapsack.
    \end{itemize}
    \pause
  \end{block}
  For the knapsack problem, we have found a brute force algorithm and a dynamic
  programming algorithm. \pause
  \begin{itemize}
    \item The brute force algorithm runs in $\Theta(2^nn)$ time. \pause
    \item The dynamic programming algorithm runs in $\Theta(nC)$ time. \pause
    \begin{itemize}
      \item There are $nC$ entries to compute. \pause
      \item For each entry, we only need $\Theta(1)$ time to get the result.
    \end{itemize}
  \end{itemize}
\end{frame}

\begin{frame}{Conclusion (cont.)}
  Is $\Theta(nC)$ considered polynomial? \pause
  No! \pause
  \begin{block}{Knapsack Problem}
    \begin{itemize}
      \item Input:
      \begin{itemize}
        \item A set of $n$ items, each with a weight $w_i$ and a value
        $v_i$.
        \item A knapsack with weight capacity $C$, represented in
        {\color{red} $m$} bits.
      \end{itemize}
      \item Output: The most valuable combination of items that fits in the
      knapsack.
    \end{itemize}
    \pause
  \end{block}
  The way to measure input size is really important. \pause
  \begin{itemize}
    \item If we use the number of bits to measure the input size, the time
    complexity of the dynamic programming method turns into
    {\color{red} $\Theta(n2^m)$}. \pause
    \item Thus, it is in fact an exponential-time algorithm. \pause
    \item Since its running time is polynomial in the numeric value of the
    input, we also say that it runs in \emph{pseudo-polynomial time}.
  \end{itemize}
\end{frame}

\section{Exercises}
\subsection{\#1}
\begin{frame}{Exercise \#1}
  We have known that the brute force algorithm runs in $\Theta(2^nn)$ time,
  while the dynamic programming algorithm runs in $\Theta(nC)$ time. \pause
  \\[1em]
  Please find the necessary and sufficient condition such that $2^nn \leq nC$.
  \\
  (In this case, the brute force algorithm does not run asymptotically slower
  than the dynamic programming algorithm.) \pause
  \begin{block}{Solution}
    The necessary and sufficient condition is $C \geq 2^n$.
  \end{block}
\end{frame}

\subsection{\#2}
\begin{frame}{Exercise \#2}
  We have introduced the knapsack problem. \pause \\[.5em]
  Now let's introduce its variation, called the unbounded knapsack problem.
  \pause
  \begin{block}{Unbounded Knapsack Problem}
    \begin{itemize}
      \item Input:
      \begin{itemize}
        \item A set of $n$ types of items, each with a weight $w_i$ and a value
        $v_i$.
        \item A knapsack with weight capacity $C$.
      \end{itemize}
      \item Output: The most valuable combination of items that fits in the
      knapsack, where the number of each type of items can be any finite
      integer.
    \end{itemize}
  \end{block}
  \pause
  Can you find any algorithm that solves the unbounded knapsack problem?
\end{frame}

\begin{frame}{Exercise \#2 (cont.)}
  \begin{block}{Solution}
    We can choose $C$ items for each type, and it turns into the
    knapsack problem with $nC$ items. \\
    Thus, the problem can be solved in $O(nC^2)$ time.
  \end{block}
\end{frame}

\subsection{\#3}
\begin{frame}{Exercise \#3}
  Another variation of knapsack problem is as follows. \pause
  \begin{block}{Fractional Knapsack Problem}
    \begin{itemize}
      \item Input:
      \begin{itemize}
        \item A set of $n$ items, each with a weight $w_i$ and a value $v_i$.
        \item A knapsack with weight capacity $C$.
      \end{itemize}
      \item Output: The most valuable combination of items that fits in the
      knapsack, where fractions of items can be taken.
    \end{itemize}
    \pause
  \end{block}
  Please find an algorithm that solves the fractional knapsack problem. \pause
  \begin{block}{Solution}
    The problem can be solved by continually choosing the most valuable items
    (according to their values per unit weight) until the knapsack is full. \\
    If one uses sorting, the problem can be solved in $O(n \log n)$ time.
  \end{block}
\end{frame}

\end{document}