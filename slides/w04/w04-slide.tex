\documentclass{beamer}
\usepackage{../mysty}

\title{Week 04: Alignment}
\author{FanFly}
\date{March 29, 2020}

\begin{document}

\begin{frame}
  \titlepage
\end{frame}

\section{Alignment}
\subsection{Edit Distance}
\begin{frame}[fragile]{Edit Distance}
  \emph{Edit distance} is a way to quantify the ``distance'' between two
  strings. \pause
  \begin{itemize}
    \item The distance is calculated by counting the minimum number of
    operations needed to transform one string into the other. \pause
    \item Different definitions of an edit distance use different sets of
    string operations.
  \end{itemize}
\end{frame}

\begin{frame}{Edit Distance (cont.)}
  Suppose that we want to find the edit distance between
  {\color{red}\lstinline{weather}}
  and
  {\color{blue}\lstinline{whether}},
  and we only allow insertion and deletion of a
  character in a string. \pause
  \begin{itemize}
    \item We can transform
    {\color{red}\lstinline{weather}}
    into
    {\color{blue}\lstinline{whether}}
    by inserting an
    {\color{blue}\lstinline{h}}
    and deleting an
    {\color{red}\lstinline{a}}, and thus the edit distance between them is 2.
    \pause
    \item Note that if we remove the different parts from both strings, we will
    get their \emph{longest common subsequence}, i.e., \lstinline{wether}.
    \pause
    \item Conversely, if we can find their longest common subsequence, we can
    calculate their edit distance.
  \end{itemize}
\end{frame}

\subsection{Longest Common Subsequence}
\begin{frame}[fragile]{Longest Common Subsequence}
  Now we introduce the longest common subsequence problem. \pause
  \begin{block}{The Longest Common Subsequence Problem}
    \begin{itemize}
      \item Input: Two sequences $A$ and $B$ of lengths $n$ and $m$,
      respectively.
      \item Output: A longest common subsequence of $A$ and $B$.
    \end{itemize}
    \pause
  \end{block}
  We can find a longest common subsequence of two sequences by finding an
  alignment between them. \pause
  \begin{center}
    \lstinline{w-eather} \\
    \lstinline{whe-ther}
  \end{center}
  \pause
  Note that there may be more than one longest common subsequences. \pause
  \begin{center}
    \lstinline{mea--surement} \qquad \lstinline{-measur-ement} \\
    \lstinline{--amus--ement} \qquad \lstinline{am---u-sement}
  \end{center}
\end{frame}

\section{Algorithms}
\subsection{Recursive Method}
\begin{frame}{Recurrence Relation}
  Let us look at some examples.
  \begin{center}
    \lstinline{c-a-ke} \quad \lstinline{-m--ask} \quad \lstinline{googol-} \\
    \lstinline{-fad-e} \quad \lstinline{i-deas-} \quad \lstinline{goog-le}
  \end{center}
  \pause
  Let $A = A' \cdot x$ and $B = B' \cdot y$, where $x$ and $y$ are the last
  elements of $A$ and $B$, respectively. \pause
  \begin{itemize}
    \item If $x = y$, then we have
    \begin{equation*}
      \text{LCS}(A, B) = \text{LCS}(A', B') \cdot x.
    \end{equation*}
    \pause
    \vspace{-1em}
    \item Otherwise, we have
    \begin{equation*}
      \text{LCS}(A, B) = \text{LCS}(A, B')
      \quad \text{or} \quad
      \text{LCS}(A, B) = \text{LCS}(A', B).
    \end{equation*}
  \end{itemize}
\end{frame}

\begin{frame}[fragile]{Recursive Method}
  Thus, it can be solved recursively! \pause
  \begin{block}{}
    \scriptsize
    \begin{lstlisting}[gobble=4]
    def lcs(A, B, n, m):
        if n == 0 or m == 0:
            return A[:0]
        elif A[n - 1] == B[m - 1]:
            return lcs(A, B, n - 1, m - 1) + A[n - 1]
        else:
            P = lcs(A, B, n - 1, m)
            Q = lcs(A, B, n, m - 1)
            if len(P) > len(Q):
                return P
            else:
                return Q
    \end{lstlisting}
  \end{block}
\end{frame}

\begin{frame}{Time Complexity of Recursive Method}
  What is its time complexity? \pause
  We have
  \begin{equation*}
    T(n, m) =
    \begin{cases}
      O(1), & \text{if $n = 0$ or $m = 0$} \\
      T(n, m - 1) + T(n - 1, m) + O(1), & \text{otherwise}.
    \end{cases}
  \end{equation*}
  \pause
  Thus, we can conclude that
  \begin{equation*}
    T(n, m) = \Theta\left(\frac{(n+m)!}{n! \cdot m!}\right),
  \end{equation*}
  which is really inefficient.
\end{frame}

\subsection{Dynamic Programming}
\begin{frame}{Dynamic Programming}
  Why the recursive method is so inefficient? \pause
  \begin{itemize}
    \item It solves the same subproblem over and over again. \pause
    \item If the algorithm can remember the solutions to the solved
    subproblems, then it can become more efficient.
    \begin{center}
      \ttfamily
      \begin{tabular}{c|cccc}
          & m & a & s & k \\ 
        \hline
        i & - & - & - & - \\
        d & - & - & - & - \\
        e & - & - & - & - \\
        a & - & a & a & a \\
        s & - & a & as & as
      \end{tabular}
    \end{center}
    \pause
    \vspace{.5em}
    \item This improvement is called \emph{dynamic programming}.
  \end{itemize}
\end{frame}

\begin{frame}{Dynamic Programming (cont.)}
  With dynamic programming, we can solve the longest common subsequence problem
  efficiently. \pause
  \begin{itemize}
    \item Assume that the length of longest common subsequence is $\ell$.
    \pause
    \item The content of each entry can be computed in $\Theta(\ell)$ time.
    \pause
    \item Thus, we can solve the problem in $O(nm\ell)$ time.
  \end{itemize}
\end{frame}

\begin{frame}{Improvement}
  In fact, the time complexity can be improved to $O(nm)$ if we only memoize
  the length of the longest common subsequence.
  \begin{center}
    \ttfamily
    \begin{tabular}{c|cccc}
        & m & a & s & k \\ 
      \hline
      i & 0 & 0 & 0 & 0 \\
      d & 0 & 0 & 0 & 0 \\
      e & \color{red}0 & 0 & 0 & 0 \\
      a & 0 & \color{red}1 & 1 & 1 \\
      s & 0 & 1 & \color{red}2 & \color{red}2
    \end{tabular}
  \end{center}
\end{frame}

\section{Exercises}
\subsection{\#1}
\begin{frame}{Exercise \#1}
  Let $A$ and $B$ be sequences of lengths $n$ and $m$, respectively. \\[.5em]
  \pause
  If the length of longest common subsequence of $A$ and $B$ is $\ell$, what is
  the edit distance between $A$ and $B$?
  \begin{itemize}
    \item Assume that only insertion and deletion are allowed to perform.
    \pause
  \end{itemize}
  \begin{block}{Solution}
    The edit distance between $A$ and $B$ is
    $(n-\ell) + (m-\ell) = n + m - 2\ell$.
  \end{block}
\end{frame}

\subsection{\#2}
\begin{frame}{Exercise \#2}
  Let $A$ be a sequence of length $n$ that does not have repeating elements. \\
  Hoe many nonempty subsequences does $A$ have? \pause
  \begin{block}{Solution}
    It has $2^n - 1$ nonempty subsequences.
  \end{block}
\end{frame}

\subsection{\#3}
\begin{frame}{Exercise \#3}
  Let $A$ and $B$ be sequences and let $f$ be a \emph{scoring function}.
  Suppose that
  \begin{equation*}
    f(x, y) =
    \begin{cases}
      p, & \text{if $x = y$} \\
      q, & \text{otherwise}.
    \end{cases}
  \end{equation*}
  \pause
  For any alignment, we can compute its score according to the scoring
  function.
\end{frame}

\begin{frame}{Exercise \#3 (cont.)}
  For example, the alignment below has score $3p + 6q$.
  \begin{center}
    \lstinline{goo---gle} \\
    \lstinline{googol---}
  \end{center}
  \pause
  Please determine the value of $p$ and $q$ such that the maximum score of
  alignment of $A$ and $B$ is equal to the length of their longest common
  subsequence. \pause
  \begin{block}{Solution}
    This property is satisfied when $p = 1$ and $q = 0$.
  \end{block}
\end{frame}

\end{document}